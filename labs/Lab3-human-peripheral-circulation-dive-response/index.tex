% Options for packages loaded elsewhere
\PassOptionsToPackage{unicode}{hyperref}
\PassOptionsToPackage{hyphens}{url}
\PassOptionsToPackage{dvipsnames,svgnames,x11names}{xcolor}
%
\documentclass[
  letterpaper,
  DIV=11,
  numbers=noendperiod,
  oneside]{scrartcl}

\usepackage{amsmath,amssymb}
\usepackage{iftex}
\ifPDFTeX
  \usepackage[T1]{fontenc}
  \usepackage[utf8]{inputenc}
  \usepackage{textcomp} % provide euro and other symbols
\else % if luatex or xetex
  \usepackage{unicode-math}
  \defaultfontfeatures{Scale=MatchLowercase}
  \defaultfontfeatures[\rmfamily]{Ligatures=TeX,Scale=1}
\fi
\usepackage{lmodern}
\ifPDFTeX\else  
    % xetex/luatex font selection
\fi
% Use upquote if available, for straight quotes in verbatim environments
\IfFileExists{upquote.sty}{\usepackage{upquote}}{}
\IfFileExists{microtype.sty}{% use microtype if available
  \usepackage[]{microtype}
  \UseMicrotypeSet[protrusion]{basicmath} % disable protrusion for tt fonts
}{}
\makeatletter
\@ifundefined{KOMAClassName}{% if non-KOMA class
  \IfFileExists{parskip.sty}{%
    \usepackage{parskip}
  }{% else
    \setlength{\parindent}{0pt}
    \setlength{\parskip}{6pt plus 2pt minus 1pt}}
}{% if KOMA class
  \KOMAoptions{parskip=half}}
\makeatother
\usepackage{xcolor}
\usepackage[left=1in,marginparwidth=2.0666666666667in,textwidth=4.1333333333333in,marginparsep=0.3in]{geometry}
\usepackage{svg}
\setlength{\emergencystretch}{3em} % prevent overfull lines
\setcounter{secnumdepth}{-\maxdimen} % remove section numbering
% Make \paragraph and \subparagraph free-standing
\ifx\paragraph\undefined\else
  \let\oldparagraph\paragraph
  \renewcommand{\paragraph}[1]{\oldparagraph{#1}\mbox{}}
\fi
\ifx\subparagraph\undefined\else
  \let\oldsubparagraph\subparagraph
  \renewcommand{\subparagraph}[1]{\oldsubparagraph{#1}\mbox{}}
\fi


\providecommand{\tightlist}{%
  \setlength{\itemsep}{0pt}\setlength{\parskip}{0pt}}\usepackage{longtable,booktabs,array}
\usepackage{calc} % for calculating minipage widths
% Correct order of tables after \paragraph or \subparagraph
\usepackage{etoolbox}
\makeatletter
\patchcmd\longtable{\par}{\if@noskipsec\mbox{}\fi\par}{}{}
\makeatother
% Allow footnotes in longtable head/foot
\IfFileExists{footnotehyper.sty}{\usepackage{footnotehyper}}{\usepackage{footnote}}
\makesavenoteenv{longtable}
\usepackage{graphicx}
\makeatletter
\def\maxwidth{\ifdim\Gin@nat@width>\linewidth\linewidth\else\Gin@nat@width\fi}
\def\maxheight{\ifdim\Gin@nat@height>\textheight\textheight\else\Gin@nat@height\fi}
\makeatother
% Scale images if necessary, so that they will not overflow the page
% margins by default, and it is still possible to overwrite the defaults
% using explicit options in \includegraphics[width, height, ...]{}
\setkeys{Gin}{width=\maxwidth,height=\maxheight,keepaspectratio}
% Set default figure placement to htbp
\makeatletter
\def\fps@figure{htbp}
\makeatother

\KOMAoption{captions}{tableheading}
\makeatletter
\@ifpackageloaded{tcolorbox}{}{\usepackage[skins,breakable]{tcolorbox}}
\@ifpackageloaded{fontawesome5}{}{\usepackage{fontawesome5}}
\definecolor{quarto-callout-color}{HTML}{909090}
\definecolor{quarto-callout-note-color}{HTML}{0758E5}
\definecolor{quarto-callout-important-color}{HTML}{CC1914}
\definecolor{quarto-callout-warning-color}{HTML}{EB9113}
\definecolor{quarto-callout-tip-color}{HTML}{00A047}
\definecolor{quarto-callout-caution-color}{HTML}{FC5300}
\definecolor{quarto-callout-color-frame}{HTML}{acacac}
\definecolor{quarto-callout-note-color-frame}{HTML}{4582ec}
\definecolor{quarto-callout-important-color-frame}{HTML}{d9534f}
\definecolor{quarto-callout-warning-color-frame}{HTML}{f0ad4e}
\definecolor{quarto-callout-tip-color-frame}{HTML}{02b875}
\definecolor{quarto-callout-caution-color-frame}{HTML}{fd7e14}
\makeatother
\makeatletter
\makeatother
\makeatletter
\makeatother
\makeatletter
\@ifpackageloaded{caption}{}{\usepackage{caption}}
\AtBeginDocument{%
\ifdefined\contentsname
  \renewcommand*\contentsname{Table of contents}
\else
  \newcommand\contentsname{Table of contents}
\fi
\ifdefined\listfigurename
  \renewcommand*\listfigurename{List of Figures}
\else
  \newcommand\listfigurename{List of Figures}
\fi
\ifdefined\listtablename
  \renewcommand*\listtablename{List of Tables}
\else
  \newcommand\listtablename{List of Tables}
\fi
\ifdefined\figurename
  \renewcommand*\figurename{Figure}
\else
  \newcommand\figurename{Figure}
\fi
\ifdefined\tablename
  \renewcommand*\tablename{Table}
\else
  \newcommand\tablename{Table}
\fi
}
\@ifpackageloaded{float}{}{\usepackage{float}}
\floatstyle{ruled}
\@ifundefined{c@chapter}{\newfloat{codelisting}{h}{lop}}{\newfloat{codelisting}{h}{lop}[chapter]}
\floatname{codelisting}{Listing}
\newcommand*\listoflistings{\listof{codelisting}{List of Listings}}
\makeatother
\makeatletter
\@ifpackageloaded{caption}{}{\usepackage{caption}}
\@ifpackageloaded{subcaption}{}{\usepackage{subcaption}}
\makeatother
\makeatletter
\@ifpackageloaded{tcolorbox}{}{\usepackage[skins,breakable]{tcolorbox}}
\makeatother
\makeatletter
\@ifundefined{shadecolor}{\definecolor{shadecolor}{rgb}{.97, .97, .97}}
\makeatother
\makeatletter
\makeatother
\makeatletter
\@ifpackageloaded{sidenotes}{}{\usepackage{sidenotes}}
\@ifpackageloaded{marginnote}{}{\usepackage{marginnote}}
\makeatother
\makeatletter
\makeatother
\ifLuaTeX
  \usepackage{selnolig}  % disable illegal ligatures
\fi
\IfFileExists{bookmark.sty}{\usepackage{bookmark}}{\usepackage{hyperref}}
\IfFileExists{xurl.sty}{\usepackage{xurl}}{} % add URL line breaks if available
\urlstyle{same} % disable monospaced font for URLs
\hypersetup{
  pdftitle={Lab 3: Peripheral Circulation and The Dive Response},
  pdfauthor={Marguerite Butler},
  colorlinks=true,
  linkcolor={blue},
  filecolor={Maroon},
  citecolor={Blue},
  urlcolor={Blue},
  pdfcreator={LaTeX via pandoc}}

\title{Lab 3: Peripheral Circulation and The Dive Response}
\author{Marguerite Butler}
\date{2023-09-01}

\begin{document}
\maketitle
\ifdefined\Shaded\renewenvironment{Shaded}{\begin{tcolorbox}[enhanced, borderline west={3pt}{0pt}{shadecolor}, boxrule=0pt, sharp corners, interior hidden, frame hidden, breakable]}{\end{tcolorbox}}\fi

\renewcommand*\contentsname{Table of contents}
{
\hypersetup{linkcolor=}
\setcounter{tocdepth}{3}
\tableofcontents
}
\hypertarget{before-lab}{%
\section{Before Lab}\label{before-lab}}

\begin{tcolorbox}[enhanced jigsaw, arc=.35mm, leftrule=.75mm, coltitle=black, left=2mm, rightrule=.15mm, colbacktitle=quarto-callout-note-color!10!white, opacitybacktitle=0.6, breakable, titlerule=0mm, colback=white, bottomrule=.15mm, toprule=.15mm, opacityback=0, colframe=quarto-callout-note-color-frame, toptitle=1mm, title=\textcolor{quarto-callout-note-color}{\faInfo}\hspace{0.5em}{Prepare for lab by:}, bottomtitle=1mm]

\begin{itemize}
\tightlist
\item
  Watch Peripheral Circulation Podcast.
\item
  Read {[}\href{../Protocols/p2-measuring-blood-pressure.pdf}{Protocol
  2}{]} on measuring blood pressure.
\item
  Read the lab manual below.\\
\item
  Write the
  {[}\href{../../labs-misc/lab-notebook.qmd\#sec-prelab}{Prelab}{]} in
  your lab notebook.
\item
  Do Quiz on Laulima (open 24 hrs before lab).
\end{itemize}

\url{https://youtu.be/h47oQH-w6F8}

\end{tcolorbox}

\begin{tcolorbox}[enhanced jigsaw, arc=.35mm, leftrule=.75mm, coltitle=black, left=2mm, rightrule=.15mm, colbacktitle=quarto-callout-note-color!10!white, opacitybacktitle=0.6, breakable, titlerule=0mm, colback=white, bottomrule=.15mm, toprule=.15mm, opacityback=0, colframe=quarto-callout-note-color-frame, toptitle=1mm, title=\textcolor{quarto-callout-note-color}{\faInfo}\hspace{0.5em}{Exercises}, bottomtitle=1mm]

\begin{itemize}
\tightlist
\item
  Measuring blood pressure
\item
  Peripheral circulation experiment

  \begin{itemize}
  \tightlist
  \item
    Develop a simple experiment to demonstrate a principle of peripheral
    circulation of choice.
  \end{itemize}
\item
  Dive response experiment

  \begin{itemize}
  \tightlist
  \item
    Develop a hypothesis for a potential trigger for the dive response.
  \end{itemize}
\end{itemize}

\end{tcolorbox}

\hypertarget{background-blood-pressure-and-peripheral-circulation}{%
\section{Background: Blood pressure and peripheral
circulation}\label{background-blood-pressure-and-peripheral-circulation}}

Vertebrates have a \textbf{closed circulatory system} where the blood is
always enclosed within blood vessels or the heart. Blood is pumped from
the \textbf{heart} (the central pump) to the \textbf{vasculature}: the
\textbf{arteries}, \textbf{capillary beds} (sites of delivery to
tissues), \textbf{the veins}, and back to heart. There are several
important consequences of this design: (1) blood pressure varies across
species according to oxygen demand and morphology (especially animal
height), (2) \textbf{blood pressure varies along the circuit}, (3)
\textbf{blood pressure can be regulated at points along the circuit},
and (4) \textbf{blood pressure can be modified situationally} depending
the state of the animal.

\hypertarget{blood-pressure-varies-across-species}{%
\subsubsection{Blood pressure varies across
species}\label{blood-pressure-varies-across-species}}

For \textbf{very active} animals (e.g., mammals and birds) or
\textbf{very large animals} (especially very tall animals that have more
gravity to resist), the ability to regulate blood pressure is critical
--- active animals will \emph{need more oxygen delivered at a faster
rate}, especially to the most metabolically active tissues, and
\emph{larger animals will require much more pressure} to reach all of
their tissues.

\hypertarget{blood-pressure-varies-during-the-cardiac-cycle.}{%
\subsubsection{Blood pressure varies during the cardiac
cycle.}\label{blood-pressure-varies-during-the-cardiac-cycle.}}

The \textbf{cardiac cycle} is a complete cycle of the heart beat,
comprised of \textbf{systole} (Figure~\ref{fig-systole}; the phase
involving contraction and ejection) and \textbf{diastole}
(Figure~\ref{fig-diastole}; relaxation and filling) of the atria and
ventricles. We will learn more about the cardiac cycle in the EKG lab.
In this lab we are focusing on the blood pressure changes. The largest
muscles of the heart are in the \textbf{ventricles}. Blood pressure is
at its highest immediately after the ventricles contract to push blood
into the arterial system (Figure~\ref{fig-wiggers}; \textbf{systolic
pressure}) and declines as the ventricles relax to fill with blood
before pumping again. Just before the ventricles contract, blood
pressure is at its lowest (\textbf{diastolic pressure}).

\begin{marginfigure}

{\centering \includesvg{index_files/mediabag/Heart_systole.svg}

}

\caption{\label{fig-systole}The human heart during the ventricular
\textbf{systole} phase of the \textbf{cardiac cycle}. Image by
\href{href=\%22https://en.wikipedia.org/wiki/User:Wapcaplet}{Wapcaplet},
\href{https://commons.wikimedia.org/wiki/User:Reytan}{Reytan},
\href{https://commons.wikimedia.org/wiki/User:Mtcv}{Mtcv} /
\href{https://commons.wikimedia.org/wiki/File:Heart_systole.svg}{Heart
systole}/\href{https://creativecommons.org/licenses/by-sa/3.0/}{CC BY-SA
3.0}}

\end{marginfigure}

\begin{figure}

{\centering \includesvg{index_files/mediabag/Heart_diasystole.svg}

}

\caption{\label{fig-diastole}The heart relaxes and the ventricles fill
during the \textbf{diastole} phase of the \textbf{cardiac cycle}. Image
by
\href{href=\%22https://en.wikipedia.org/wiki/User:Wapcaplet}{Wapcaplet},
\href{https://commons.wikimedia.org/wiki/User:Reytan}{Reytan},
Vector:\href{https://commons.wikimedia.org/wiki/User:Sjef}{Sjef} /
\href{https://commons.wikimedia.org/wiki/File:Heart_diasystole.svg}{Heart
diastole}/ \href{https://creativecommons.org/licenses/by-sa/3.0/}{CC
BY-SA 3.0}}

\end{figure}

\begin{figure}

{\centering \includesvg{index_files/mediabag/Wiggers_Diagram_2.svg}

}

\caption{\label{fig-wiggers}Volume and pressure changes during the
\textbf{cardiac cycle}, as shown in a Wiggers diagram. Note that aortic
and ventricular pressures are both lowest and the end of diastole, just
before the beginning of systole. adh30 revised work by DanielChangMD who
revised original work of DestinyQx; Redrawn as SVG by xavax,
\href{https://commons.wikimedia.org/wiki/File:Wiggers_Diagram_2.svg}{Wiggers
Diagram 2},
\href{https://creativecommons.org/licenses/by-sa/4.0/legalcode}{CC BY-SA
4.0}}

\end{figure}

\hypertarget{blood-pressure-varies-along-the-vascular-circuit.}{%
\subsubsection{Blood pressure varies along the vascular
circuit.}\label{blood-pressure-varies-along-the-vascular-circuit.}}

Blood in the arteries leaving the heart is always at very high pressure
as compared to the low pressure in the veins in the legs or the even
lower pressure in capillary beds at the tissues. Blood pressure drops as
the blood vessels branch again and again, increasing the cross-sectional
area of the circuit, until it reaches the capillaries where the tissues
experience relatively constant, low pressure to facilitate
\textbf{diffusion}.

The slow blood flow at the capillaries facilitates diffusion of oxygen,
nutrients, and carbon dioxide and other wastes between the blood and the
tissues that are bathed by the capillaries. Therefore, \textbf{pressure
varies} depending on \textbf{distance from the heart}, the
\textbf{cross-sectional area of the blood vessels}, as well as
\textbf{gravity}. However, at any given point along the circuit, blood
pressure remains fairly constant.

\hypertarget{circulation-can-be-adjusted-situationally.}{%
\subsubsection{Circulation can be adjusted
situationally.}\label{circulation-can-be-adjusted-situationally.}}

At most times, blood pressure is regulated to \textbf{maintain a
relatively constant pressure}, however, there are times when
\textbf{circulation needs to be adjusted}. A well-known example is the
\textbf{Fight-or-Flight response}, which occurs, for example, when an
animal sees a predator or anticipates a fight. The \textbf{sympathetic
nervous system} dominates and causes a ramp-up of circulation to deliver
more energy to the skeletal muscles: increased \textbf{cardiac output}
(= \textbf{heart rate} x \textbf{stroke volume}) and \textbf{blood
pressure}, and increased blood flow to the lungs and skeletal muscles.
In contrast, the \textbf{rest-and-digest} response occurs after an
animal has had a large meal. The \textbf{parasympathetic nervous system}
dominates, lowering heart rate, concentrates blood flow to the gut, and
promotes a resting state.

Adjustments to blood flow are not simply an adjustment of heart
function, but also \textbf{constriction or relaxation of the
vasculature} (blood vessels: arteries, veins, capillaries).
\textbf{Constricting blood vessels} will reduce their
\textbf{cross-sectional area} and \textbf{increase blood pressure and
flow}.

Local changes in circulation are under \textbf{nervous} and
\textbf{hormonal} control. Regulation of blood flow in the vertebrate
circulatory system occurs by three primary mechanisms: 1) \textbf{local
receptors} (\emph{nervous system}) to detect levels of metabolic
activity (e.g., carbon dioxide receptors), 2) \textbf{sympathetic} and
\textbf{parasympathetic} (\emph{autonomic nervous system}) control of
the vasculature including capillary beds at the tissues, and 3)
\textbf{endocrine} (\emph{hormonal}) control of the vasculature.

In this lab, we will measure blood pressure of volunteers using a finger
pulse transducer, a stethoscope, a blood pressure cuff
(sphygnomanometer), and changes in peripheral circulation by measuring
the volume of the extremities using a belt with a force transducer. We
will do a series of learning exercises and then conduct an experiment on
factors affecting peripheral circulation and as well as during simulated
dives (the dive response).

\hypertarget{equiptment}{%
\subsection{Equiptment}\label{equiptment}}

\begin{itemize}
\tightlist
\item
  PowerLab data acquisition system
\item
  Finger pulse transducer
\item
  Stethoscope
\item
  Blood pressure cuff
\item
  Blood pressure gauge (sphygnomanometer) with pod or BNC port
\item
  Respiratory belt transducer
\item
  LabChart software, note Blood Pressure settings
\end{itemize}

\hypertarget{exercise-1-measuring-blood-pressure}{%
\section{Exercise 1: Measuring Blood
Pressure}\label{exercise-1-measuring-blood-pressure}}

Traditionally, systemic arterial blood pressure is measured using a
\textbf{stethoscope} and a \textbf{blood pressure cuff} connected to a
blood pressure gauge called a \textbf{sphygnomanometer}
(\emph{sss-fig-no-ma-nom-eter}). The sphygnomanometer is calibrated in
pressure units of mmHg (millimeters of mercury). Modern instruments use
compressed air as a hydraulic fluid to transmit the force of your
pulsing blood.

Refer to {[}\href{../Protocols/p2-measuring-blood-pressure.pdf}{Protocol
2.1 and 2.2}{]} for how to measure blood pressure.

\hypertarget{setup}{%
\subsection{Setup}\label{setup}}

\begin{enumerate}
\def\labelenumi{\arabic{enumi}.}
\tightlist
\item
  Use ``Blood Pressure'' settings to start Chart software.
\item
  Setup Finger Pulse transducer on Input 1.
\item
  Attach sphygmomanometer transducer to Input 2 (pod input).
\end{enumerate}

\hypertarget{data-collection}{%
\subsection{Data Collection}\label{data-collection}}

\begin{enumerate}
\def\labelenumi{\arabic{enumi}.}
\tightlist
\item
  Measure blood pressure on a human volunteer using
  \textbf{auscultation} (listening through a stethoscope) and a
  \textbf{sphygnamonometer} .
\item
  Measure blood pressure using the PowerLab system and LabChart. Check
  that the \textbf{channel settings} are correctly set for each channel.
\item
  Repeat (1) and (2) on each group member, making sure to comment your
  data trace.
\end{enumerate}

\hypertarget{questions-for-thought}{%
\subsubsection{Questions for
thought\ldots{}}\label{questions-for-thought}}

\begin{enumerate}
\def\labelenumi{\arabic{enumi}.}
\tightlist
\item
  Does the time of the first \textbf{Korotkoff sound} (systolic pressure
  heard through the stethoscope), correspond with the first appearance
  of blood flow (as measured by the finger pulse)? Why or why not?
\item
  Would slowing the rate of pressure release from the cuff make your
  readings of the first appearance of blood flow more accurate? What
  problems might be caused by slowing pressure release?
\item
  Does the time that diastolic pressure is heard through the stethoscope
  correspond with anything particular in the blood flow signal? Can you,
  therefore, use pulse measurement to replace the stethoscope?
\item
  How much variation in measurement of diastolic and systolic pressures
  was observed within and between individuals? What are potential
  sources of variation in these estimates?
\end{enumerate}

\hypertarget{exercise-2-exploring-peripheral-circulation}{%
\section{Exercise 2: Exploring Peripheral
Circulation}\label{exercise-2-exploring-peripheral-circulation}}

\hypertarget{objectives}{%
\subsection{Objectives}\label{objectives}}

To demonstrate basic principles of peripheral circulation using blood
pressure data from the extremities. What you would expect based on
relative distance from the heart and gravity (and whether the location
is above or below the heart)?

\hypertarget{procedure}{%
\subsection{Procedure}\label{procedure}}

\begin{enumerate}
\def\labelenumi{\arabic{enumi}.}
\tightlist
\item
  \emph{Brainstorm with your lab group to develop some simple
  experiments to demonstrate principles of peripheral circulation.} What
  are some good hypotheses for peripheral blood pressure?
\item
  What are some good locations to measure (or other simple
  manipulations) for comparison? Make sure you place the stethoscope on
  a major artery or vein such as the radial artery on the forearm, or
  the small saphenous vein on the calf. Ask for help if you don't know
  where they are. Be specific when you write up your methods or we will
  not understand what you did.
\item
  For each experiment, \textbf{determine both systolic and diastolic
  blood pressure}.
\end{enumerate}

\hypertarget{notes}{%
\subsubsection{Notes}\label{notes}}

\begin{enumerate}
\def\labelenumi{\arabic{enumi}.}
\tightlist
\item
  You may need to recalibrate the blood pressure force transducer after
  each time you move the cuff.
\item
  Place the instruments directly on the skin (not through clothes).\\
\item
  When measuring from foot, please wash toe before attaching pulse
  transducer to prevent any fungal contamination.
\item
  \textbf{Always Release the cuff pressure \emph{completely} as soon as
  you are done taking data}
\end{enumerate}

\hypertarget{analysis}{%
\subsubsection{Analysis}\label{analysis}}

Compare systolic and diastolic pressure for each of your treatments
versus an appropriate control. Think carefully about appropriate
controls for your ideas to achieve the best test of your hypotheses.

\hypertarget{questions-for-thought-1}{%
\subsubsection{Questions for
thought\ldots{}}\label{questions-for-thought-1}}

\begin{enumerate}
\def\labelenumi{\arabic{enumi}.}
\tightlist
\item
  How much does blood pressure change for each treatment? What could
  explain it? Does it seam reasonable?
\item
  How much variation is there among members of your group? What are
  sources of variation in these estimates?
\end{enumerate}

\hypertarget{exercise-3-the-dive-response}{%
\section{Exercise 3: The Dive
Response}\label{exercise-3-the-dive-response}}

When an air-breathing animal dives, it voluntarily holds its breath
while the tissues continue to use oxygen. The \textbf{dive response} is
a reflexive response that reorganizes circulation to maintain blood flow
to the most essential organs -- the brain, eyes, and myocardium (heart
muscle), while reducing blood flow to the peripheral tissues including
musculature of the limbs and thorax, lungs, and renal system.
Remarkably, all vertebrates have a dive response. The responses vary
greatly between taxa, with some of the most pronounced being in whales
and diving seals.

A primary feature of the dive response is a \textbf{diving bradycardia}
(slowing of the heart rate), which results in a dramatic drop in
\textbf{cardiac output} (\textbf{cardiac output} = \textbf{heart rate x
stroke volume}). You might think that this would result in a dramatic
drop in blood pressure as well, but in addition to reduced cardiac
output, another component of the dive response is \textbf{peripheral
vasoconstriction}, where the blood vessels of the peripheral tissues are
constricted or even closed. As a whole, the dive response preserves
circulation around the vital organs while reducing circulation to the
peripheral tissues. Oxygen becomes depleted and carbon dioxide and
lactate build up in the tissues during a dive. When the animal
resurfaces, there is a recovery period characterized by more rapid heart
rate and ventilation to absorb more oxygen and flush out lactate and
carbon dioxide.

\hypertarget{objectives-1}{%
\subsection{Objectives}\label{objectives-1}}

You will investigate the effects of the diving response on heart rate
and peripheral circulation in humans during simulated dives. First, you
will examine the effect of holding your breath, then you will examine
the effects of simulated dives. Finally, you will do a series of
experiments to determine which stimuli contribute to triggering the dive
response. One person will serve as the experimental subject. If time
permits, each person in the group should take turns being the
experimental subject to have more replication.

\hypertarget{additional-required-equipment}{%
\subsection{Additional Required
Equipment}\label{additional-required-equipment}}

\begin{itemize}
\tightlist
\item
  Respiratory Belt Transducer
\item
  Wash basin, Ice, Thermometer
\item
  Duct tape
\item
  Use the Dive settings file
\end{itemize}

\hypertarget{sec-divesetup}{%
\subsection{A. Set up}\label{sec-divesetup}}

\begin{marginfigure}

{\centering \includegraphics{../../images/calf_belt.jpg}

}

\caption{\label{fig-calf}Attachment of the respiratory belt transducer
to the calf for leg volume measurement.}

\end{marginfigure}

\begin{enumerate}
\def\labelenumi{\arabic{enumi}.}
\tightlist
\item
  Use the Exercise 1 setup with the addition of the Respiratory Belt
  Transducer to input 3 (or 2) to measure leg volume. Check the channel
  settings to make sure they match the inputs. Ask your TA for the
  proper settings.
\item
  Attach the respiratory belt snugly to the calf
  (Figure~\ref{fig-calf}). \emph{It should feel tight and the sensor
  fabric should be slightly stretched.}
\item
  Place the sphygnomanometer cuff around the subject's thigh, and
  \emph{duct tape it securely so that it can be pressurized to restrict
  blood flow}. Be sure to apply tape to \emph{secure both the top and
  the bottom} of the cuff.
\item
  Record for 10 seconds and stop. Scale the Pulse channel and the Leg
  Volume channel to fully display the data.
\item
  Record again and test by flexing and relaxing your calf. \emph{You
  should be able to see a clear deflection on the leg volume channel.}
  If it is very small, try tightening the respiratory belt a little.
  Check with your TA before moving on.
\item
  Use a timer to time the treatments (a cell phone or a web browser will
  do).
\end{enumerate}

\begin{tcolorbox}[enhanced jigsaw, arc=.35mm, leftrule=.75mm, coltitle=black, left=2mm, rightrule=.15mm, colbacktitle=quarto-callout-note-color!10!white, opacitybacktitle=0.6, breakable, titlerule=0mm, colback=white, bottomrule=.15mm, toprule=.15mm, opacityback=0, colframe=quarto-callout-note-color-frame, toptitle=1mm, title=\textcolor{quarto-callout-note-color}{\faInfo}\hspace{0.5em}{The idea behind measuring peripheral circulation using leg volume
changes}, bottomtitle=1mm]

We can quantify the volume in your peripheral circulation (specifically
your lower leg) by assessing \textbf{venous pooling} for a standard time
interval. By constricting blood flow to the lower limb, we will prevent
venous return of the blood. Because the veins do not have much smooth
muscle, it is relatively easy to stop venous return.

You will use the sphygnomanometer cuff to cut off circulation in the leg
for 30 sec.~at the upper thigh. The respiratory belt transducer senses
stretch and can be used to measure \textbf{calf volume}
(Figure~\ref{fig-calf}) \textbf{before} inflating the cuff,
\textbf{after 30 sec of inflation}, and \textbf{after deflating} the
cuff.

\textbf{The resulting \emph{leg volume} difference} (maximum-minimum
volume at T1 vs.~T2; Figure~\ref{fig-legvol}) is a measure of pooling
and therefore peripheral circulation. For each experiment, we will
compare the \textbf{degree of pooling} at \textbf{rest}, during the
\textbf{experimental treatment}, and during the \textbf{recovery}
periods.

\end{tcolorbox}

\begin{marginfigure}

{\centering \includegraphics{../../images/legvol.jpg}

}

\caption{\label{fig-legvol}Zoom window view of measuring the leg volume
change resulting from a simulated dive using a marker at T1 (30sec of
cuff inflation) and the waveform cursor at T2 (maximum leg volume drop
after releasing the pressure).}

\end{marginfigure}

\begin{tcolorbox}[enhanced jigsaw, arc=.35mm, leftrule=.75mm, coltitle=black, left=2mm, rightrule=.15mm, colbacktitle=quarto-callout-tip-color!10!white, opacitybacktitle=0.6, breakable, titlerule=0mm, colback=white, bottomrule=.15mm, toprule=.15mm, opacityback=0, colframe=quarto-callout-tip-color-frame, toptitle=1mm, title=\textcolor{quarto-callout-tip-color}{\faLightbulb}\hspace{0.5em}{Measuring leg volume from the volume trace:}, bottomtitle=1mm]

\begin{itemize}
\tightlist
\item
  Set the Marker to a region just prior to cuff deflation in the leg
  volume channel (Figure~\ref{fig-legvol}).
\item
  Use the Waveform Cursor to determine the relative change in leg volume
  for the data trace when the cuff was deflated.
\end{itemize}

\end{tcolorbox}

\begin{tcolorbox}[enhanced jigsaw, arc=.35mm, leftrule=.75mm, coltitle=black, left=2mm, rightrule=.15mm, colbacktitle=quarto-callout-tip-color!10!white, opacitybacktitle=0.6, breakable, titlerule=0mm, colback=white, bottomrule=.15mm, toprule=.15mm, opacityback=0, colframe=quarto-callout-tip-color-frame, toptitle=1mm, title=\textcolor{quarto-callout-tip-color}{\faLightbulb}\hspace{0.5em}{Protocol: Leg volume measurement}, bottomtitle=1mm]

\begin{itemize}
\tightlist
\item
  Record the subject's resting pulse for 15 seconds.
\item
  \textbf{Inflate} the cuff to \textbf{80 mmHg} (or whatever pressure
  feels tight enough to restrict blood flow for the subject),
\item
  \textbf{Hold pressure} for \textbf{exactly 30 seconds} (\emph{NOTE:
  You may have to gently squeeze the bulb to keep pressure at 80 mmHg.})
\item
  \textbf{Quickly and COMPLETELY release} the pressure
  (Figure~\ref{fig-legvol}).
\item
  Record until the volume returns to baseline.
\end{itemize}

{NOTES: Comment at \textbf{start}, at \textbf{start to inflate}, at
\textbf{pressure}, and at \textbf{deflate}.}\\
When doing repeated measurements, ensure you have \textbf{baseline data}
for \emph{at least 15 sec} before inflating the cuff again. The leg
volume measurement will take a little over a minute total.

\end{tcolorbox}

\hypertarget{b.-breath-holding-experiment}{%
\subsection{B. Breath holding
experiment}\label{b.-breath-holding-experiment}}

\begin{enumerate}
\def\labelenumi{\arabic{enumi}.}
\tightlist
\item
  Use the \href{@sec-divesetup}{Section A setup} with the respiratory
  belt on the calf (Figure~\ref{fig-calf}) and the sphygnomanometer cuff
  on the upper thigh.
\item
  Measure \textbf{leg volume at rest} {[}\href{@prot-legvol}{Protocol:
  Leg volume measurement}{]}.\\
\item
  \textbf{Breath hold treatment} (will require breath hold of
  \textasciitilde1min):

  \begin{enumerate}
  \def\labelenumii{\alph{enumii}.}
  \tightlist
  \item
    Have the subject take one or two deep breaths, exhale partially, and
    then hold their breath while they place their head down on the lab
    bench.\\
  \item
    Start a timer with the start of breath hold (cell phone or find a
    timer online). \emph{The experimenter should tap on the subjectʻs
    shoulder every 10 sec to help them track time.}
  \item
    Measure \textbf{leg volume 30 seconds into the breath-hold}.\\
  \end{enumerate}
\item
  Measure \textbf{leg volume during recovery}: at 30 sec into the
  recovery.
\end{enumerate}

{Make sure to \textbf{comment at each step} and \textbf{DEFLATE CUFF
COMPLETELY} each time.}

\hypertarget{c.-dive-response-experiment}{%
\subsection{C. Dive response
experiment}\label{c.-dive-response-experiment}}

Fill your bucket with icewater deep enough to submerge your face up to
your temples. Use a thermometer to monitor temperature. \emph{Note: The
receptors that trigger the dive response are in the temples, so it is
critical that the temples be submerged in order to see the dive
response.}

\begin{enumerate}
\def\labelenumi{\arabic{enumi}.}
\tightlist
\item
  \href{@sec-divesetup}{Section A setup}.
\item
  Before beginning, allow the subject to find a comforable chair height
  and leg posture to allow them to remain as motionless as possible
  throughout the experiment. Most people sit, but standing is OK if
  preferred.
\item
  \textbf{Resting:}

  \begin{enumerate}
  \def\labelenumii{\alph{enumii}.}
  \tightlist
  \item
    Have the subject position their face just above the water's surface,
    and remain motionless.
  \item
    Perform a \textbf{leg volume measurement}. (reminder: record the
    baseline, while restricting blood flow, and recovery.)
  \end{enumerate}
\item
  \textbf{Simulated Dive:}

  \begin{enumerate}
  \def\labelenumii{\alph{enumii}.}
  \tightlist
  \item
    If you feel the need to, practice a simulated dive and recovery with
    a dry run (without submerging face) before conducting the
    experiment.)
  \item
    While recording, have the subject take a deep breath, exhale
    partially, and then hold their breath while immersing their face up
    to their temples in the pan of water for 30 sec.~
  \item
    Perform a \textbf{leg volume measurement} 30 sec into the simulated
    dive.
  \item
    At the end of the simulated dive, signal to the subject to
    resurface.
  \end{enumerate}
\item
  \textbf{Recovery:} Perform a \textbf{leg volume measurement} 30 sec
  into recovery.\\
\item
  Continue to record for another 15 seconds.
\end{enumerate}

\begin{tcolorbox}[enhanced jigsaw, arc=.35mm, leftrule=.75mm, coltitle=black, left=2mm, rightrule=.15mm, colbacktitle=quarto-callout-note-color!10!white, opacitybacktitle=0.6, breakable, titlerule=0mm, colback=white, bottomrule=.15mm, toprule=.15mm, opacityback=0, colframe=quarto-callout-note-color-frame, toptitle=1mm, title=\textcolor{quarto-callout-note-color}{\faInfo}\hspace{0.5em}{NOTES:}, bottomtitle=1mm]

\begin{itemize}
\tightlist
\item
  \textbf{Make sure everything is very clear before beginning} to avoid
  repeating this experiment.
\item
  One member of the group should tap the subject on the back at
  10-second intervals while immersed to help them keep track of the time
  and prevent anxiety.
\item
  Work out in advance what your signals will be for timing
  vs.~resurfacing.
\item
  Make \textbf{good comments} and \textbf{minimize movement} in the
  Finger Pulse Transducer.
\item
  \emph{Do not force the subject to remain submerged}.\\
\end{itemize}

\end{tcolorbox}

\hypertarget{d.-additional-experiment}{%
\subsection{D. Additional Experiment}\label{d.-additional-experiment}}

The experiments above provide a basic demonstration of the dive
response. However, the experiment involved multiple stimuli
simultaneously. \textbf{Brainstorm} how you might \emph{identify the
components which are actually ``triggering'' the dive response by
isolating stimuli.} Are these components all necessary? Are they
additive?

Each group should \textbf{design an experiment to isolate one potential
stimulus} responsible for triggering the dive response and perform it.
Get your idea approved by your TA. Share your results with the other
groups. \emph{Make sure you explain your methods carefully (including
your logic) in your lab report.}

\hypertarget{analysis-1}{%
\subsection{Analysis}\label{analysis-1}}

\begin{enumerate}
\def\labelenumi{\arabic{enumi}.}
\tightlist
\item
  Compare the heart rate, pulse volume, and leg volumes before, during
  and after a breath-hold.
\item
  Do the same for your simulated dive as well as for your additional
  experiment.
\end{enumerate}

\hypertarget{questions-for-thought-.-.-.}{%
\subsection{Questions for thought . .
.}\label{questions-for-thought-.-.-.}}

\begin{enumerate}
\def\labelenumi{\arabic{enumi}.}
\tightlist
\item
  Compare your results of heart rate during breath holding with those
  from simulated dives. Are they the same?
\item
  What factors could explain differences between breath holding and a
  ``dive''? Have you eliminated any hypotheses with your experiments?
\item
  Compare the percent change in heart rate during dives among the
  members of your group. Is the relative bradycardia similar? Or is the
  absolute difference similar? What comparisons can you make between
  different treatments to dig deeper into your numbers?
\item
  Do your results for leg volume suggest that peripheral circulation
  changes during a breath-hold? What specific result points to this?
\item
  Did your peripheral circulation increase or decrease during a
  ``dive''?
\item
  Why do you think the diving response is considered advantageous?
\end{enumerate}

\hypertarget{after-lab-assignment-week-3}{%
\section{After Lab: Assignment Week
3:}\label{after-lab-assignment-week-3}}

\begin{itemize}
\tightlist
\item
  You will work with your lab group to analyze data, and you may share
  figures if you wish. However, each person will submit an
  \textbf{Individual WorkSheet} {[}\href{Lab3ws.qmd}{html}{]}
\item
  Reminder: \emph{Practical has been moved to next week (week 4) on Lab
  1 material}. Let us know if you want to come in to practice.
\end{itemize}



\end{document}
